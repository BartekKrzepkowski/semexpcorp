\documentclass{beamer}
\usetheme{simple}
\usepackage{lmodern}
\usepackage[scale=2]{ccicons}
\usepackage{polski}	
\usepackage[polish]{babel}	
\usepackage{listings}
\usepackage{listingsutf8}
\usepackage{graphicx}
\usepackage{smartdiagram}
\usepackage[utf8]{inputenc}
\usepackage[T1]{fontenc}
\usepackage{tabularx}

\frenchspacing

%\usepackage[backend=bibtex, citestyle=authoryear-icomp]{biblatex}
%\addbibresource{bibliografia}
	

\title{Informacje o korpusach}
\subtitle{Czyli czym będziemy się zajmować na zajęciach}
\date{\today}
\author{Bartosz Maćkiewicz}
\institute{Instytut Filozofii, Uniwersytet Warszawski}

\begin{document}

\maketitle

\begin{frame}
  \frametitle{Jakie korpusy będziemy omawiać?}
  \Large
  \begin{itemize}
  \item Narodowy Korpus Języka Polskiego
  \item British National Corpus
  \item Corpus of Contemporary American English
  \end{itemize}
\end{frame}

\begin{frame}
  \frametitle{Z jakich narzędzi będziemy korzystać?}
  \Large
  \begin{itemize}
  \item wyszukiwarki korpusowej \textbf{Pelcra} oraz kolokatora, która umożliwia pracę z NKJP;
  \item wyszukiwarki korpusowej \textbf{Poliqarp}, która umożliwia prace z NKJP;
  \item multiwyszukiwarki i kolokatora \textbf{corpus.byu.edu} umożliwiającej pracę z COCA, BNC i wieloma innymi dostepnymi korpusami
  \item programu do zarzadzania i przetwarzania korpusów \textbf{SketchEngine}
  \end{itemize}
\end{frame}

\begin{frame}
  \Large
  \frametitle{Jak tworzone są korpusy?}
  \begin{itemize}
  \item próba vs całe teksty
  \item zróżnicowanie vs różnorodność
  \item anotacja
  \item oportunizm
  \end{itemize}
\end{frame}

\begin{frame}
  \Large
  \frametitle{Struktura tekstowa korpusów}
  \textbf{reprezentatywność vs zrównoważenie}
  \begin{itemize}
   \item \textbf{reprezentatywność} - odnoszenie się do jakiejś rzeczywistości istniejącej poza korpusem
   \item \textbf{zrównoważenie} - dbałość o taką budowę korpusu, żeby żaden składnik na żadnym z poziomów nie dominował nad innym.
  \end{itemize}
\end{frame}

\begin{frame}
  \large
  \frametitle{Struktura tekstowa w NKJP}
  Górski i Łaziński (2012) podają następujące proporcje tekstów:
  \begin{itemize}
  \item Publicystyka i krótkie wiadomości prasowe: 50%
  \item Literatura piękna: 16%
  \item Literatura faktu: 5,5%
  \item Typ informacyjno-poradnikowy: 5,5%
  \item Typ naukowo-dydaktyczny: 2,0%
  \item Inne teksty pisane: 3,0%
  \item Książka niebeletrystyczna nieklasyfikowana: 1,0%
  \item Teksty konwersacyjne, mówione medialne i quasi-mówione razem: 10,0%
  \item Teksty internetowe statyczne i dynamiczne razem: 7,0%
  \end{itemize}
\end{frame}

\begin{frame}
  \frametitle{Jakie informacje są dostępne w korpusach?}
  \vspace{0.5em}
  {\Large \textbf{metadane}}
  \vspace{0.5em}

  ,,Aby zdjąć ze mnie ten straszny obowiązek, ten rozkaz piekielny, ksiądz zabije innego człowieka''
  {\footnotesize
  \begin{itemize}
\item \textbf{autor}: Jarosław Iwaszkiewicz
\item \textbf{tytuł}: Brzezina i inne opowiadania Kościół w Skaryszewie
\item \textbf{źródło}: Brzezina i inne opowiadania Kościół w Skaryszewie
\item \textbf{ISBN}: 9788307030838
\item \textbf{rok publikacji}: 2006
\item \textbf{wydawca}: Czytelnik
\item \textbf{miejsce publikacji}: Warszawa
\item \textbf{typ}: literatura piękna
\item \textbf{kanał}: książka
  \end{itemize}
  }

\end{frame}

\begin{frame}
  \frametitle{Jakie informacje są dostępne w korpusach?}
  \vspace{0.5em}
  {\Large \textbf{anotacje}}
  \vspace{0.5em}

  ,,Kto miał wg Ciebie obowiązek informowania opinii publicznej o tej sprawie?''

  \vspace{0.5em}

\textbf{[informować:ger:sg:gen:n:imperf:aff]}
  \begin{itemize}
  \item forma bazowa tego słowa to "informować" (lemat)
  \item klasa gramatyczna tego słowa to rzeczownik odczasownikowy (ger)
  \item jest to rzeczywnik w liczbie pojedynczej (sg)
  \item przypadek tego rzeczownika to dopełniacz (gen)
  \item rodzaj tego rzecownika to rodzaj nijaki (n)
  \item czasownik, od którego derywowana jest ta forma jest czasownikiem niedokonanym (imperf)
  \item jest to forma niezanegowana (czyli nie jest to np. /niepoinformowanie/) (aff)
  \end{itemize}
\end{frame}

% \begin{frame}
%   \frametitle{Bibliografia}
%   \printbibliography 
% \end{frame}


 \end{document}